\documentclass[pdftex,11pt,a4paper]{moderncv}

% moderncv themes
%\moderncvtheme[blue]{casual}
\moderncvtheme[blue]{classic}

% character encoding
\usepackage[utf8]{inputenc}

% adjust the page margins
\usepackage[scale=0.8]{geometry}
%\setlength{\hintscolumnwidth}{3cm}                      % if you want to change the width of the column with the dates
%\AtBeginDocument{\setlength{\maketitlenamewidth}{6cm}}  % only for the classic theme, if you want to change the width of your name placeholder (toA leave more space for your address details
%\AtBeginDocument{\recomputelengths}                     % required when changes are made to page layout lengths

% personal data
\firstname{Mauro}
\familyname{Tubiana}
\address{Via Caravaggio, 3}{35020, Albignasego (PD), Italia}
\mobile{(+39) 347 9372741}
%\phone{phone (optional)}
%\fax{fax (optional)}
\email{mauro.tubiana@gmail.com}
%\photo[64pt]{picture}  % '64pt' is the height the picture must be resized to and 'picture' is the name of the picture file
%\quote{Some quote (optional)}

% to show numerical labels in the bibliography; only useful if you make citations in your resume
\makeatletter
\renewcommand*{\bibliographyitemlabel}{\@biblabel{\arabic{enumiv}}}
\makeatother

% bibliography with mutiple entries
%\usepackage{multibib}
%\newcites{book,misc}{{Books},{Others}}

%----------------------------------------------------------------------------------
%            content
%----------------------------------------------------------------------------------
\begin{document}
\sffamily
\maketitle

\section{Educazione}

\cventry{2008--2011}{Laurea Magistrale}{Universit\`a degli Studi di Padova}{}{\textit{102/110}}{
Laurea Magistrale in Ingegneria Informatica
}
\cventry{2004--2008}{Laurea Triennale}{Universit\`a degli Studi di Padova}{}{\textit{96/110}}{
Laurea Triennale in Ingegneria Informatica
}
\cventry{1999--2004}{Diploma}{ITIS Galileo Galilei}{Conegliano}{}{
Diploma di \textit{Perito Industriale Capotecnico Elettronica e Telecomunicazioni}
}

\section{Tesi Magistrale}

\cvline{Titolo}{\emph{Segmentazione congiunta di immagini ed informazione geometrica e suo utilizzo all’interno
di algoritmi di segmentazione stereoscopica}}
\cvline{Relatore}{Prof. Pietro Zanuttigh}
\cvline{Correlatore}{Ing. Carlo Dalmutto}

\section{Esperienze Lavorative}

\cventry{2021--}{Assegnista}{Dipartimento di Scienze Chimiche}{Padova}{}{
\begin{itemize}
	\item Sviluppo di un sensore colorimetrico per la stima del pH
	\item Principali mansioni svolte:
		\begin{enumerate}
			\item Scelta di componenti hardware
			\item Implementazione di software per automatizzazione di alcune attivit\`a
			\item Implementazione di software per la gestione del dispositivo, acquisizione,
			elaborazione delle immagini	e stima del pH
			\item Supporto alla progettazione meccanica del distositivo
			\item Implementazione di tool a sviluppo della ricerca
		\end{enumerate}
\end{itemize}
}
\cventry{2020--2021}{Professore a contratto}{Dipartimento di Ingegneria dell'Informazione}{Padova}{}{
\begin{itemize}
	\item Titolo del corso: \emph{Embedded Real Time Control}, impartito in inglese
	\item Corso concernente l'implementazione di sistemi di controllo in tempo reale su microcontrollore
	\item Attivit\`a di laboratorio riguardante la realizzazione di sistemi di controllo utilizzando un rover ad-hoc
	\item Il corso \`e suddiviso nei seguenti moduli:
		\begin{enumerate}
			\item Sistemi embedded
			\item Programmazione concorrente
			\item Sistemi real-time
		\end{enumerate}
\end{itemize}
}

\cventry{2016-2020}{Sviluppatore Firmware e Software}{DigitalLighting}{Padova}{}{
\begin{itemize}
	\item Personalizzazione driver per sistemi Linux
	\item Personalizzazione OperWRT
	\item Sviluppo di tool per la produzione
	\item Personalizzazione di bootloader su Cortex-M0
	\item Sviluppo di firmware utilizzando tecnologie BluetoothLE e Bluetooth Mesh
\end{itemize}
}

\cventry{2014-2016}{Sviluppatore Firmware e Software}{WearIT}{Padova}{}{
\begin{itemize}
	\item Sviluppo applicazioni per sistema operativo Android
	\item Personalizzazione di driver, principalmente per touchscreen
\end{itemize}
}

\cventry{2012-2014}{Sviluppatore Firmware}{Si14 S.p.A.}{Padova - Marghera}{}{
\begin{itemize}
	\item Porting di Kernel Linux, principalmente su SoC ARM
	\item Sviluppo driver user-space su sistemi Linux-based (GNU ed Android)
	\item Porting Android su hardware personalizzato
	\item Personalizzazione di distribuzioni embedded
	\item Sviluppo firmware su microcontrollori e SoC Atmel e Nordic
	\item Sviluppo di tool per l’automatizzazione di test hardware
\end{itemize}
}

\section{Lingue}
\cvlanguage{Italiano}{}{Madrelingua}
\cvlanguage{Inglese}{Scritto}{Buono}
\cvlanguage{}{Parlato}{Buono}

\section{Competenze informatiche}
\cvline{OS}{Buona conoscenza di sistemi operitivi GNU/Linux, in paricolare Debian-based, uso di sistemi operativi Microsoft}
\cvline{Linguaggi}{Buona conoscenza di C/C++, Perl, Matlab, \LaTeX, scripting BASH, Java, conoscenza base
di Html, Ruby, Rust, Haskell e altri linguaggi funzionali, Zig, Nim}
\cvline{Controllo di versione}{Buona conoscenza di git, minima conoscenza di mercurial ed SVN}
\cvline{Ambienti di sviluppo}{Basilare esperienza con Visual Studio, VSCode e Keil, buona conoscenza di gcc, gdb, make, vim,
Eclipse, Android Studio, STM32CubeIDE ed Embedded Studio. Buona conoscenza dell'"ecosistema" Arduino, conoscenza basilare dell'ambiente
di sviluppo ESP32}
\cvline{Librerie e Framework}{Buona conoscenza di OpenCV (interfaccia C e C++), con particolare attenzione al modulo
core, esperienze con OpenGL, OpenMP ed OpenMPI su piattaforma AIX}
\cvline{Kernel Linux}{Conoscenza di alcuni aspetti relativi allo sviluppo di driver e codice kernel-space su kernel
2.6.x, 3.x e 4.x}
\cvline{Android}{Porting su sistemi custom e sviluppo applicazioni}
\cvline{RTOS}{Utilizzo di sistemi operativi real time su microcontrollore (FreeRTOS) per implementazione di sistemi di controllo}

\section{Competenze tecniche}
\cvline{Grafica}{Minima esperienza nell’uso di Flash, Photoshop e Gimp}
\cvline{Elettronica}{Basi di elettronica digitale}
\cvline{SoC}{Freescale iMX6 (ARM CortexA9), Atheros AR9331 (MIPS), Texas Instruments AM335 (ARM CortexA8), Freescale i.MX28 (ARM 9), Nordic NRF5x
(ARM Cortex M0 con Bluetooth LE), Atmel SAMD21 (ARM Cortex-M0), ESP32}
\cvline{Schede di sviluppo}{Schede basate su ESP32, Arduino, RaspberryPI}
\cvline{Microcontrollori}{Atmel AVR 8 bit, Atmel 8051, STM32L4xx, STM32F7xx}
\cvline{Radio}{Bluetooth LE e mesh, minima esperienza con transceiver radio sub-Ghz, GPS, wifi mesh}
%\cvcomputer{category 1}{XXX, YYY, ZZZ}{category 4}{XXX, YYY, ZZZ}

\renewcommand{\listitemsymbol}{-} % change the symbol for lists

%\section{Others}
%\cvlistitem{Italian B driving licence}
%\cvlistdoubleitem{Item 3}{}
\section{}
\emph{Autorizzo il trattamento dei dati da me forniti ai sensi della legge 675/96 sulla privacy.}

\end{document}
